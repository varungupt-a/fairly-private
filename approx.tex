% !TEX root = main.tex

\section{Approximate fairness with differentially privacy}



We now turn to the finite sample setting where we release a hypothesis
that minimizes the training error. Our goal is now approximate
fairness. We use the exponential mechanism in the same way as it is
used in private PAC learning \cite{Kasiviswanathan:2011:WLP:2078965.2078976}.
Defining the sample as $Z$, $|Z| = n$ and
$\Gamma^{Z}$ as our in-sample fairness measure, we give the algorithm:
$$\mathcal{A}^\eps : \text{Output hypothesis }h \in \mathcal{H} \text{
	with probability proportional to }$$
\begin{align}
\exp(-\frac{\eps \cdot u(Z,h)}{2\Delta u})
\end{align}

where

$$u(Z,h) = \Gamma^Z(h) + err^Z(h)$$.
$$\Delta u(Z,h) = \Delta\ell + \Delta{\Gamma} \approx O(1/n+1/|\z{10}|+1/|Z_{11})$$

$$err^Z(h) = \frac{1}{n} \sum_{(x,y) \in Z}\Pr[h(x) \neq y]$$



This algorithm is the exponential mechanism in \citet{McSherry:2007:MDV:1333875.1334185}, and so it is differentially private.

\begin{lemma}
	The algorithm $\A_\eps$ is $\eps$-differentially private.
\end{lemma}

Note that except for when $|\H|$ is polynomial, the exponential
mechanism does not necessarily yield a polynomial time algorithm.


\begin{theorem}
	(Generic private fair learner) For all $d \in \mathbb{N}$, any
	concept class $\cC_d$ whose cardinality is at most
	$\exp(\text{poly}(d))$ is privately and approximately fairly
	agnostically learnable using $\H_d = \cC_d$. More precisely, the
	learner uses $n = ..$ labeled examples from $D$, where $\eps,
	\alpha$, and $\beta$ are parameters of the private learning.
\end{theorem}

\begin{proof}
	Let $\A_{\eps}$ be as defined above. The privacy condition is
	satisfied by Lemma.

	Now we show that the utility condition is also satisfied. Let the
	event $E = \{\A_{\eps} = h \text{ with } u(h) > OPT + \alpha\}$.
	We need that $\Pr[E] \leq \beta$ to get that with probability $\geq 1-\beta$,

	$$u(h) \leq OPT + \alpha$$
	$$\Rightarrow \Gamma(h) + err(h) \leq OPT_\Gamma + OPT_{err} +\alpha$$
	$$\Rightarrow \Gamma(h) \leq OPT_\Gamma +\alpha \text{ and } err(h) \leq OPT_{err} +\alpha$$

	Recall that $Z$ is the sample drawn i.i.d. from a distribution $D$.
	By Chernoff-Hoeffding bounds and \cite{woodworthFollowUp}, (insert appendix ref. see below proof
	for now)

	$$\Pr[|u(Z,h) - u(D,h)| \geq \rho] \leq 18\exp(-\min_{ya}{\frac{\rho^2 n P_{ya}}{16}})$$

	for all hypotheses $h \in \H_d$. Hence by union bound,

	$$\Pr[|u(Z,h) - u(D,h)| \geq \rho \text{ for some } h \in \H_d] \leq 18|\H_d|\exp(-\min_{ya}{\frac{\rho^2 n P_{ya}}{16}})$$

	Now we analyze $\A_\eps(Z)$ conditioned on the event that for all
	$h\in \H_d$, $|u(Z,h) - u(D,h)| < \rho$. For every $h \in \H_d$, $\Pr[\A_\eps(Z) = h]$ is

	$$\frac{\exp(-\frac{\eps}{2\Delta u} \cdot
		u(Z,h))}{\sum_{h'\in\H_d}\exp(-\frac{\eps}{2\Delta u} \cdot u(Z,h'))}
	\leq \frac{\exp(-\frac{\eps}{2\Delta u} \cdot
		u(Z,h))}{\max_{h'\in\H_d}\exp(-\frac{\eps}{2\Delta u} \cdot u(Z,h'))} $$
	$$= \exp(-\frac{\eps}{2\Delta u}(u(Z,h) - \min_{h'\in\H_d}u(Z,h')))$$
	$$\leq \exp(-\frac{\eps}{2\Delta u}(u(Z,h) - (OPT + \rho)))$$

	Hence the probability that $\A_\eps(Z)$ outputs a hypothesis $h \in
	\H_d$ such that $u(Z,h) > OPT + 2\rho$ is at most
	$|\H_d|\exp(-\frac{\eps\cdot\rho}{2\Delta u})$

	Setting $\rho = \alpha/3$. If $u(D,h) \geq OPT + \alpha$ then
	$|u(D,h) - u(Z,h)| \geq \alpha/3$ or $u(Z,h) \geq OPT + 2\alpha/3$.
	Hence

	$$\Pr[E] \leq |\H_d|(18\exp(-\min_{ya}{\frac{\rho^2 n P_{ya}}{16}}) + \exp(-\frac{\eps\cdot\alpha}{6\Delta u})) \leq \beta$$.

	Where the inequality holds for $n \geq $.
\end{proof}
\begin{theorem}[Real-valued Additive Chernoff-Hoeffding Bound.]
	Let $X_1,...,X_d$ be i.i.d. random variables with $\mathbb{E}[X_i] = \mu$ and $a \leq X_i \leq b$ for all $i$. Then for every $\rho > 0$,
	$$Pr[|\frac{\sum_i X_i}{n} - \mu| > \rho] \leq 2\exp(\frac{-2\rho^2n}{(b-a)^2})$$
\end{theorem}

\begin{lemma}\text{\cite{woodworthFollowUp}} For $\delta \in (0,1/2)$ and a binary predictor $h$,
	$$\Pr[|\Gamma(h) - \Gamma^Z(h)|>\rho ] \leq 16 \exp(\frac{1}{4}\rho^2n
	\min_{ya}{P_{ya}}) = \delta$$
\end{lemma}

\begin{lemma}[Concentration of utility]
	For $\delta \in (0,1/2)$ and a binary predictor $h$,
	$$\Pr[|u(Z,h) - u(D,h)| > \rho] \leq 18\exp(-\min_{ya}{\frac{\rho^2 n
	P_{ya}}{16}}) = \delta$$
\end{lemma}
\begin{proof}
	Fix an $h \in \H$. Let $err^Z = err^Z(h)$, $\Gamma^Z = \Gamma^Z(h)$, $err = err(h)$, $\Gamma = \Gamma(h)$ and $P_{ya} = \Pr[Y=y, A=a]$
	\begin{align*}
	\Pr[|u(Z,h) - u(D,h)| > \rho] & = \Pr[|\Gamma^Z + err^Z - (\Gamma +err) | > \rho]\\
	&\leq \Pr[|\Gamma^Z - \Gamma|  + |err^Z - err| > \rho]\\
	&\leq \Pr[|\Gamma^Z - \Gamma|> \frac{\rho}{2}]  + \Pr[|err^Z - err| > \frac{\rho}{2}]\\
	&\leq 16\exp(-\min_{ya}{\frac{\rho^2 n P_{ya}}{16}}) + 2\exp(\frac{-2\rho^2n}{4})\\
	&\leq 18\exp(-\min_{ya}{\frac{\rho^2 n P_{ya}}{16}})\\
	\end{align*}

	Where the first inequality follows from the triangle inequality, the second from union bound, and the third from Lemma 4.4 and Theorem 4.3.
\end{proof}


\dk {Note: can do extension into other definitions of fairness (with constant factor).
	Can also extend into a different norm for $u(Z,h)$ if we have a
	concentration of measure theorem for the different norm. There are
	also random matrix concentration bounds (can do concentration for all
	functions of the confusion matrix?)}
